% Instituto Federal de Educação, Ciência e Tecnologia Baiano - Campus Guanambi
% 
% Modelo para Trabalho de Conclusão de Curso em LaTeX
% Superior de Análise e Desenvolvimento de Sistemas
% Alterado por: Dr. Naidson Clayr Santos Ferreira
%
% ----------------------------------------------------------------------- %
% Arquivo: resumo.tex
% ----------------------------------------------------------------------- %

% RESUMO--------------------------------------------------------------------------------

\begin{resumo}[RESUMO]
\begin{SingleSpacing}

% Não altere esta seção do texto--------------------------------------------------------
\imprimirautorcitacao. \imprimirtitulo. \imprimirdata. \pageref {LastPage} f. \imprimirprojeto\ – \imprimirprograma, \imprimirinstituicao. \imprimirlocal, \imprimirdata.\\
%---------------------------------------------------------------------------------------

Este trabalho apresenta um estudo sobre o Pacificador, o sistema de comando e controle desenvolvido pelo Exército, que tem como principal objetivo a garantia do sucesso das mais variadas operações desenvolvidas pelo Exército e apoiadas por este sistema. 
Na seção 2 realizaremos a definição de Comando e Controle, abordando de forma separada tanto o Comando como o Controle, suas características e os fatores determinantes no sucesso de uma operação. Falaremos também sobre os componentes imprescindíveis do Comando e Controle e suas interdependências.
Abordaremos na seção 3 o Sistema de Comando e Controle, sua principal missão, as categorias de informações mais importantes, funções básicas de um sistema C2, os cinco subconjuntos básicos para o sistema C2, a importância do recurso humano bem treinado para realizar as entradas fidedignas no sistema e a importância da computação no uso dos sistemas de Comando e Controle.
Na próxima seção apresentaremos as principais funcionalidades do Sistema Pacificador Web, quais órgãos do governo utilizam-no e em quais grandes eventos o sistema foi utilizado como ferramenta de comando e controle, bem como suas especificidades que foram determinantes para o sucesso daqueles eventos. Também falaremos sobre o impacto da mobilidade ou da falta dela para bom cumprimento da missão pelos agentes, o que é determinante para o pacificador.
Na última seção, realizaremos a análise estatística do uso do Sistema Pacificador Web pelos oito Comandos Militares de Área (C Mil A), iniciando pela distribuição e frequência de Centro de Operações pelos diversos C Mil A, distribuição de frequência de número de Agentes por cada grande comando, quantidade de Incidentes, Ações e Relatos nos diversos Comandos Militares e por último faremos uma conclusão da análise desses dados.
\\

\textbf{Palavras-chave}: comando e controle, grandes eventos, sistema de comando e controle, Pacificador.

\end{SingleSpacing}
\end{resumo}

% OBSERVAÇÕES---------------------------------------------------------------------------
% Altere o texto inserindo o Resumo do seu trabalho.
% Escolha de 3 a 5 palavras ou termos que descrevam bem o seu trabalho 
