% Instituto Federal de Educação, Ciência e Tecnologia Baiano - Campus Guanambi
% 
% Modelo para Trabalho de Conclusão de Curso em LaTeX
% Superior de Análise e Desenvolvimento de Sistemas
% Alterado por: Dr. Naidson Clayr Santos Ferreira
%
% ----------------------------------------------------------------------- %
% Arquivo: resumo.tex
% ----------------------------------------------------------------------- %

% RESUMO--------------------------------------------------------------------------------

\begin{resumo}[RESUMO]
\begin{SingleSpacing}

% Não altere esta seção do texto--------------------------------------------------------
\imprimirautorcitacao. \imprimirtitulo. \imprimirdata. \pageref {LastPage} f. \imprimirprojeto\ – \imprimirprograma, \imprimirinstituicao. \imprimirlocal, \imprimirdata.\\
%---------------------------------------------------------------------------------------

Este trabalho apresenta um estudo com a utilização de mineração de dados voltada ao contexto educacional, em cursos superiores de informática, onde há um elevado número de alunos reprovados nos componentes curriculares de programação. A necessidade de sua realização surgiu a partir da observação de uma performance insatisfatória de estudantes com relação às disciplinas de algoritmo e estrutura de dados. Para tanto, escolheu-se a utilização de dados de alunos, matriculados em uma instituição pública de ensino superior, a serem coletados e analisados utilizando-se de algoritmos de Máquina de Vetor de Suporte e processados com a ferramenta WEKA, a fim de se criar modelos de regressão capazes de tanto apontar riscos de evasão quanto predizer as possíveis causas do elevado número de reprovações, em tais disciplinas.\\

\textbf{Palavras-chave}: Mineração de Dados Educacionais. Evasão Escolar. Maquina de Vetor de Suporte.

\end{SingleSpacing}
\end{resumo}

% OBSERVAÇÕES---------------------------------------------------------------------------
% Altere o texto inserindo o Resumo do seu trabalho.
% Escolha de 3 a 5 palavras ou termos que descrevam bem o seu trabalho 
