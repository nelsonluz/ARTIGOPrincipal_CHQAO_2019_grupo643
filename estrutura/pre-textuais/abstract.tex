% Instituto Federal de Educação, Ciência e Tecnologia Baiano - Campus Guanambi
% 
% Modelo para Trabalho de Conclusão de Curso em LaTeX
% Superior de Análise e Desenvolvimento de Sistemas
% Alterado por: Dr. Naidson Clayr Santos Ferreira
%
% ----------------------------------------------------------------------- %
% Arquivo: abstract.tex
% ----------------------------------------------------------------------- %

% ABSTRACT--------------------------------------------------------------------------------

\begin{resumo}[ABSTRACT]
\begin{SingleSpacing}

% Não altere esta seção do texto--------------------------------------------------------
\imprimirautorcitacao. \imprimirtitleabstract. \imprimirdata. \pageref {LastPage} f. \imprimirprojeto\ – \imprimirprograma, \imprimirinstituicao. \imprimirlocal, \imprimirdata.\\
%---------------------------------------------------------------------------------------

This work presents a study with the use of data mining focused on the educational context, in higher education courses, where there is a high number of students who fail in the curricular components of programming. The need for its realization arose from the observation of an unsatisfactory student performance in relation to the disciplines of algorithm and data structure. In order to do so, we chose the use of data from students enrolled in a public institution of higher learning, to be collected and analyzed using Support Vector Machine algorithms and processed with the WEKA tool, in order to create regression models capable of both pointing out evasion risks and predicting the possible causes of the high number of failures in such disciplines.\\

\textbf{Keywords}:Educational Data Mining. School Evasion. Support Vector Machine.

\end{SingleSpacing}
\end{resumo}

% OBSERVAÇÕES---------------------------------------------------------------------------
% Altere o texto inserindo o Abstract do seu trabalho.
% Escolha de 3 a 5 palavras ou termos que descrevam bem o seu trabalho 
