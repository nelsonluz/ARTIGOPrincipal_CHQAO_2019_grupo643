% Instituto Federal de Educação, Ciência e Tecnologia Baiano - Campus Guanambi
% 
% Modelo para Trabalho de Conclusão de Curso em LaTeX
% Superior de Análise e Desenvolvimento de Sistemas
% Alterado por: Dr. Naidson Clayr Santos Ferreira
%
% ----------------------------------------------------------------------- %
% Arquivo: justificativa.tex
% ----------------------------------------------------------------------- %

% JUSTIFICATIVA-----------------------------------------------------------------

\chapter{JUSTIFICATIVA}
\label{chap:justificativa}
O pacificador é uma ferramenta relativamente nova no EB e que é testada cada vez que acontece algum grande evento, ou seja, é na prática que ele é testado e aprovado ou reprovado em alguns quesitos, dessa forma, acreditamos que realizar um levantamento estatístico ajudaria a sua equipe de desenvolvedores entender alguns gargalos e prováveis falhas que poderiam ser acertadas, assim como, confirmar tudo aquilo que tem funcionado muito bem ao longo desses anos.
Esta maturidade somente pode ser obtida, com a confirmação que suas funcionalidades estão atendendo perfeitamente às necessidades da Força Terrestre e de todos os órgãos envolvidos nos mais variados eventos em que o pacificador é utilizado, e isso no nosso entender avaliza uma análise qualitativa e quantitativa sobre suas funcionalidades, quais solicitações são mais comuns, quais solicitações ficam sem ser atendidas ou atendidas em parte, quais os logs de erros mais comuns na ferramenta.

Devidamente autorizados a ter acesso aos dados salvos em seus servidores e com o único fim de realizar este estudo, e sem identificar cada grande evento, estudaremos cada área que entendermos ser importante relatar e coletaremos a informação proveitosa à nossa pesquisa, sendo assim, acreditamos que ao longo do estudo devam surgir outros questionamentos ou até informações que podem ser de grande relevância ao que está proposto nesse tópico.

% No Brasil, segundo \citeonline{gomes2008proposta}, \citeonline{doslinguagem} e \citeonline{junior2005ensino}, as disciplinas de algoritmo e programação são responsáveis tanto por um grande número de reprovações como também representam uma das causas da evasão de estudantes em cursos de computação no ensino superior e médio/técnico em instituições públicas. Tal fato pode estar relacionado ao nível teórico do material estudado uma vez que a programação possui conceitos próprios e, de certa forma, pouco compartilhados com as demais disciplinas, sua especificidade nos módulos iniciais bem como a não difusão do contexto nas escolas públicas de ensino básico nos anos anteriores.

% De acordo com \citeonline{manhaes2011previsao}, diversos fatores podem ser apontados como possíveis causas de evasão de alunos, entre eles é possível reconhecer aspectos socioeconômicos regionais, localização geográfica, tempo de duração e adequação ao mercado de trabalho. Isso significa que suas principais causas podem variar entre instituições e regiões específicas e, por se tratar de um componente curricular de significância maior dentro de um curso de computação, as causas de seu elevado número de reprovações e evasões devem ser melhor estudadas e analisadas para que se possa criar indicadores capazes de determinar as principais fontes do problema.
%Na Justificativa deve-se mostrar, com fatos e argumentos, porque o projeto é importante e deve ser desenvolvido. Quais fatos, ideias ou leituras mostram que o tema precisa ser estudado? Qual a relevância do tema? Também é necessário pensar de qual Linha de Pesquisa do TC o projeto estaria mais próximo.

%Para tal, discussão mais aprofundada se torna necessária a fim de demonstrar como isso pode vir a atingir o ensino, não apenas em determinada instituição mas, também, dentro de um contexto social. É necessário que se discuta suas causas e consequências e como elas se relacionam para que, até mesmo, seus professores tenham melhor ideia de como lidar com esse quesito, sem que haja perda no desempenho dos alunos ou até dos próprios professores, beneficiando a instituição e sociedade ao final desse processo. Por outro lado, negar a existência do problema em questão, ignorando sua origem e efeitos, bem como a importância da discussão do tema poderá representar um esgotamento no curso de computação com rapidez proporcional ao nível de evasão dos estudantes.

%Portanto, um estudo aprofundado no quesito é necessário para que sejam analisadas as dificuldades envolvidas no ensino de programação em certas instituições, por se tratar de um problema de interesse comum, o que poderá, até mesmo, contribuir com processos que englobam temas socioeconômicos regionais inseridos na realidade de cada instituição, além de apresentar relevante valor para o meio acadêmico. 

