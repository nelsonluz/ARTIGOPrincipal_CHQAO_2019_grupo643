\chapter{SISTEMA DE COMANDO E CONTROLE}
% \hspace{1.5cm}
Segundo \cite{thesecommandandcontrol} a  principal missão do sistema de Comando e Controle (C2) é atender as necessidades dos comandantes. O sistema deve proporcionar aos comandantes total uso de todos os recursos de forma efetiva e eficiente do emprego das forças militares por toda a extensão longitudinal dos conflitos. As três categorias de informação que são associadas ao sistema de C2 são: status dos amigos, status dos inimigos e status do ambiente operacional. Todos os sistemas de C2 devem ser capazes de executar as seguintes funções básicas: coleta de dados; exibição dos processos; disseminação; e manutenção de dados a respeito da informação. O \cite{comandoecontrole2015} define sistema de Comando e Controle como: 
\begin{quote}
    Conjunto de instalações, equipamentos, comunicações, doutrina, procedimentos e pessoal essencial para o comandamento, em nível nacional, das crises e dos conflitos.
\end{quote}

% \hspace{1.5cm}
A definição oficial, indica cinco subconjuntos básicos, para o sistema de C2: comunicações; pessoal;  doutrina; procedimento; instalações. A comunicações é o subsistema de entrada no sistema de C2, mas não é necessariamente o mais importante. Devido as  dificuldades dos Exércitos modernos na gerência dos longos campos de batalha é requerido um capaz sistema de comunicações, para um efetivo e eficiente controle das forças. Os avanços tecnológicos das comunicações a partir da 2ª Guerra Mundial, tem permitidos aos comandantes a manutenção de exercitar o C2 sobre as tropas. A ligação entre todos os componentes dos sistemas de C2 e os comandantes é realizados através das comunicações.

% \hspace{1.5cm}
O pessoal é supostamente um significativo componente do sistema, haja vista que o homem é ao mesmo tempo a parte mais complexa e frágil. Em um sistema de C2 o homem fornece muitas entradas em todos os níveis. Estas entradas têm efeito direto no processo de decisão individual dos comandantes e nas suas percepções das informações  sobre os dados recebidos. O mau treinamento destes indivíduos ou sua afetação por ações de combate faz com que as informações contidas nos sistemas de C2 não sejam adequadamente úteis para os comandantes. Com o advento da tecnologia de inteligência artificial, é possível a criação de um sistema utilizado para diminuir ou remover as baixas humanas no campo de batalha, porém não é um problema fácil de se resolver. A habilidade de reportar um problema por um subordinado, pode oferecer uma grande diferença na guerra, não sendo facilmente substituindo por um sistema automatizado.

% \hspace{1.5cm}
Outro importante subconjunto do sistema de C2 são os equipamentos e instalações, somente aqueles que não fazem parte das comunicações e pessoal. Neles são incluídos os sensores, os computadores, os equipamentos de exibição, bem como, os locais onde será realizado a operação e manutenção dos equipamentos. Os computadores passaram a ser a principal peça dentro do processo de comando e controle. Devido a evolução tecnológica dos computadores dentro sistema de C2 passou a ser tão significativo que foi solicitado a mudança do termo C2 para C4 (acrescentado Comunicações e Computação). Computadores apresentam performance nas seguintes funções: sensores e comunicações de redes; correlação, filtragem e análise de informação sobre o inimigo; manutenção dos status e localização das forças amigas; viabilidade e estabelecimento dos planos desenvolvimento; e evolução dos planos de batalha e seus engajamentos. Os computadores aumentam a capacidade dos comandantes na coleta e processamento de grande quantidade de dados, que facilita a sua capacidade de decisão. Mas, o comandante pode ficar saturado com a quantidade de informação, por isto é importante a utilização de um aplicativo com dados precisos, adequados, confiáveis e usualmente válido. A influência do homem tem impacto significativo no sistema de C2 na utilização da aplicação computacional. O grande desafio para a comunidade de comando e controle é a operação de sistema de interoperável e no futuro a alta capacidades dos sistema de C2.

% \hspace{1.5cm}
Dentro de um sistema de C2 está incluído todos os procedimentos usados para o planejamento, direção, coordenação e controle das forças na efetuação de uma determinada missão. Estes procedimentos são publicados pelo comandante que é o responsável pela performance das tarefas iniciadas e os pré-determinados padrões operacionais. Os modernos sistemas de C2 são complexos e constantemente cheio de tecnologia que são fundamental para função de comando e controle.

% \hspace{1.5cm}
O entendimento da definição de Comando e Controle e os sistemas de C2, são a chave para entender as complexas estruturas organizacionais e operacionais deste sistema. O sistema Pacificador será descrito, como o sistema de C2 utilizado em grandes eventos e operações de GLO no nível estratégico. 
