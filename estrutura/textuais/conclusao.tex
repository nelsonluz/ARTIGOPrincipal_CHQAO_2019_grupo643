% Instituto Federal de Educação, Ciência e Tecnologia Baiano - Campus Guanambi
% 
% Modelo para Trabalho de Conclusão de Curso em LaTeX
% Superior de Análise e Desenvolvimento de Sistemas
% Alterado por: Dr. Naidson Clayr Santos Ferreira
%
% ----------------------------------------------------------------------- %
% Arquivo: conclusao.tex
% ----------------------------------------------------------------------- %

% CONCLUSÃO--------------------------------------------------------------------

\chapter{CONCLUSÃO}
\label{chap:conclusao}

O Exército Brasileiro tem um papel importante nos mais variados cenários do Estado Brasileiro, seja na ajuda humanitária da operação pipa no Nordeste, ou nas catástrofes como as que vem acontecendo ao longo do ano de 2019, as quais podemos citar operação contra incêndios na Amazônia brasileira e limpeza do óleo nas praias da Região Nordeste, além disso, podemos citar intervenção federal no Estado do Rio de Janeiro, entre tantas outras operações realizadas pela força terrestre em território nacional e até mesmo internacional nas operações de paz.
Vivemos um momento sem precedentes na história da segurança pública, onde o crime organizado se vale de tecnologia para realizar o planejamento de suas ações, além de ter armamentos e equipamentos com alto poder de dissuasão. É neste contexto que o Pacificador Web é inserido, com o objetivo de apoiar à decisão dos comandantes das mais variadas operações trazendo todo tipo de informação relevante sobre o terreno, status dos amigos e inimigos entre tantas informações necessárias para o cumprimento da missão. 
Entendemos que o Pacificador Web é uma excelente ferramenta que se adequa a qualquer uma das operações que o Exército executa no momento, tanto que já foi testado e aprovado inúmeras vezes nas mais variadas operações desde a Conferência das Nações Unidas (Rio +20) em 2012 até as operações que estão sendo executadas neste momento.
Vimos que um importante “gargalo” na utilização do Pacificador é a dependência pela telefonia móvel, que é gerenciada por empresas privadas de telefonia e que nem sempre têm uma cobertura de sinal satisfatória nas localidades onde as operações se desenvolvem. Dessa maneira, entendemos que poderia ser encontrado um meio mais seguro de interligação entre os dispositivos móveis e os servidores, como uma rede via satélite.
Analisando os dados no período compreendido entre janeiro de 2017 até outubro de 2019, vimos que a maioria dos Centros de Operações estão lotados no Comando Militar do Leste (CML) com mais de 32 por cento da totalidade de todo o território nacional, sendo que o Comando Militar do Sudeste (CMSE) é o que tem menos Centros de Operações com 4 por cento. Isto comprova que o CML no momento é o C Mil A que mais utiliza o Pacificador, além de ser o Comando que mais tem operações de GLO, no período escolhido para análise.
Na avaliação que fizemos no número de Agentes por C Mil A, o CML ainda pontua como o que tem mais agentes com mais de 14 mil agentes, enquanto o CMS que é considerado o maior C Mil A tem apenas 350 agentes no período da análise.
Já na tabela 3 identificamos um aumento de incidentes significativo no CMA e CMNE o que leva a crermos que o Pacificador foi amplamente utilizado na Operação Acolhida e nas operações de combate ao fogo na Amazônia e também nas operações de limpeza do óleo nas praias nordestinas.
O Pacificador é uma realidade no âmbito do Exército, uma ferramenta consolidada que somente precisa manter-se atualizada frente às peculiaridades de cada tipo de operação desenvolvida pelo Exército, dito isso, acreditamos que uma futura pesquisa de campo junto aos agentes que estão nos mais variados teatros de operações, com o intuito da realização de um levantamento de quais funções e atividades precisam ser melhoradas para que este sistema fique o mais adequado o possível, sendo assim citamos uma melhoria que seria a entrada de incidentes e todo tipo de dado seja facilitada pelo a habilitação de reconhecimento de voz o que traria mais segurança aos agentes responsáveis pela entrada de dados.


