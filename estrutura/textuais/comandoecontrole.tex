\chapter{COMANDO E CONTROLE}

Segundo \cite{undestanding2006}, a função Comando e Controle não é um fim em si mesmo, porém é meio para consegui criar valores (ex: o rastreamento de uma operação. Basicamente, C2 é sobre evidenciar esforços sobre determinados números de entidades, sendo elas Organizações ou pessoas, recursos, tais como informação gerando assim algumas tarefas, objetivos e metas. A definição de Comando e Controle é  incompleta e potencialmente inútil a menos que venha a prover medidas existentes. 


Mesmo o C2 sendo necessário ele não é a garantia do sucesso da operações. Este sucesso tem dependência de vários outros fatores, incluindo a disponibilidade dos meios apropriados, suas capacidades operacionais e além dos inimigos e outras adversidades do Teatro de Operação. O êxito de Comando e Controle não pode ser definido pelo o sucesso nas missões, levando em conta que missão é uma ação de comando.
% Talvez tenha que recuar pois copie todo o texto do EB20-MC-10.205
A definição de Comando e Controle encontrada no Manual EB20-MC-10.205 é:
\begin{quote}
 Constitui-se  no  exercício  da  autoridade  e  da  direção que um comandante tem sobre as forças sob o próprio comando, para o cumprimento da missão  designada.  Viabiliza  a  coordenação  entre  a  emissão  de  ordens  e  diretrizes  e  a  obtenção de informações sobre a evolução da situação e das ações desencadeadas. \cite{comandoecontrole2015}
\end{quote}
Esta concisa e limpa definição de C2 e suas terminologias estabelecem o alicerce para o entendimento de como se dá o Comando e Controle. 


Conforme  \cite{comandoecontrole2015} \textit{Comando} representa o poder revestido de autoridade de comando dentro de uma ação/operação militar sobre os seus subordinados através de sua patente, em uma determinada tarefa. Já \textit{Controle}  é definido como a direção que o Comandante da a sua tropa e seu controle operacional. Independentemente de seu emprego ser colocado erroneamente, à medida que o comandante busca impor sua vontade no campo de batalha é imperativo que o controle porte-se em favor do comando. O triunfo em uma operação se dará pelo o efetivo adestramento em C2 por uma considerável tropa. Igualmente, uma força sem controle sobre os seus meios, pessoal e processos pode ser levada a derrota no teatro operacional. 


O Comando e Controle é coberto por três componentes  imprescindível e interdependentes, que são:
\begin{itemize}
    \item a autoridade, legalmente investida, de onde provém as decisões que materializam o exercício do comando, que influenciara no exercício de controle com o fluxo de informações.
    \item o processo decisório, baseado no arcabouço doutrinário, através da criação das ordens e determinado o caminho que deve seguir as informações até tua concretização.
    \item a estrutura, que inclui pessoal, instalações, equipamento e tecnologias que auxiliam na atividade de comando e controle.
\end{itemize}
Este componentes ajudam a estabelecer a relação de comando com a intenção de mostrar ao comandante a dimensão em que sua autoridade é exercida, no processo hierárquico entre subordinados e superiores. No bloco subsequente conversaremos sobre a teoria de sistema de comando e controle. 
